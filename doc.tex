\section{Introduction}

\section{Box: Kappa tutorial}

\section{Results}

Demographic stochasticity can generate ``emergent'' heterogeneity even in the absence of explicit differences between individual-based rates.  In simple models, this heterogeneity can be characterized. We explicate the notion that this is predictable (see Box). \jd{Is that really what Box is doing, though? Or more about linking the emergent stochasticity in the deterministic vs.\ demographic-stochastic models?}


\begin{figure}
\centering
\includegraphics{lsFig.Rout.pdf}
\caption{
	\textbf{Heterogeneity emerges even from a simple, linearized compartmental model} due to implicit variation in recovery times among infectors. (left) Activity distributions (density curves) and secondary case distributions (density histograms) for the outset of an SIR epidemic. Because the first bin (at zero) sits at the boundary of support for each distribution, we have plotted this bin as double the density and half the width; this adjustment preserves area-to-area correspondence with the PDF, while facilitating visual comparison of the heights of the density and mass functions. (right) Inequality curves for \emph{activity} distributions from SIR models with differing \Ro~are identical (and indistinguishable due to overplotting); inequality in the \emph{case} distribution decreases with \Ro~towards the theoretical limit of the activity distribution.
}\flab{ls}
\end{figure}

\Fref{ls} shows patterns of emergent heterogeneity in the early stage of an outbreak.

But despite differences in a non-dynamic world, we find invariance in case-per-case when looking across the entire epidemic. 
The top panel of \Fref{RcAverage} shows realized distributions of ``offspring cases'' caused by individual infectors across a simple, stochastic SIR epidemic. The distributions remain indistinguishable across a wide range of the key parameter \Ro.
This seems surprising. 
The resolution is that larger epidemics with larger \Ro\ have larger between-cohort variation, as expected, but that is balanced by smaller within-cohort variation (\Fref{RcAverage}, lower panel). 
\begin{figure}
\centering
\includegraphics{RcAverage.Rout.pdf}
\caption{
\textbf{Nearly identical empirical distributions of case-per-case for epidemics of different strengths conditioned over the entire outbreak cycle.}
	Epidemic outbreaks have been simulated for a population of $10000$ with different \Ro\, and the number of secondary cases per infector has been recorded at the time of outbreak extinction (top panels).
	Identical variance in case-per-case for epidemics of different strengths modeled using simple compartmental models (lower panel).
	As \Ro~increases, between-cohort variance rises, and within-cohort variance falls, maintaining constant total variance of 1.
}\flab{RcAverage}
\end{figure}

\begin{figure}
\centering
\includegraphics{kappa/RcTimePlotVaryingPeak.Rout.pdf}
\caption{
\textbf{Nowcasting the case-per-case distribution evolves as the outbreak unfolds.}
 Panel a: Incidence represents the size of each cohort infected at each rescaled time point.
By ``rescaled,'' we mean that time has first been scaled by the mean infectious period and is then measured relative to the outbreak peak.
Panel b: At each time, the mean and standard deviation of cases caused by each case are calculated using only cohorts infected up to that time.
The mean cases per case is larger in early-infected cohorts and depends on the key parameter \Ro. 
As time goes by, later-infected cohorts have larger weights, as measured by incidence. These cohorts experience susceptible pool depletion, leading to fewer secondary cases and a lower mean case-per-case number. 
After the peak, the susceptible population size stabilizes, and the mean and variance approach 1. 
Panel c: The between-cohort component of the squared coefficient of variation $\kappa_{bet}$ is negligible among the early-infected cohorts, as they experience a similar size of susceptibles. As the outbreak unfolds, the decline in the susceptible population drives between-cohort variation, which is more pronounced in stronger epidemics. The squared coefficient of variation $\kappa$ reaches its minimum around the outbreak peak. The largest cohort, measured by incidence, experiences a sharp depletion of susceptibles, and the impact of variation in recovery time fades—declining the variance and, in turn, $\kappa$. After the peak, $\kappa$ rebounds to one. 
In panel c, the squared coefficient of variation is shown, split into within-group and between-group components. 
For panels b and c, the y-axis value at each time point was computed using only the cohorts that had been infected up to that time.
}\flab{timeCutoff}
\end{figure}


We are also interested in what emergent distributions will look like to people studying outbreaks in real time. We are interested, at least to some extent, both in how cohorts change through time, and in what the outbreak will ``look like'' if we observe from a particular time.  \Fref{timeCutoff} is one example; we are working on others.

2025 Dec 09 (Tue) suggestion for figures. We want:

\begin{itemize}

\item a naive truncated figure that assigns to each cohort the number of actual cases up until a particular time.\ag{Does \fref{Naive} do what we are looking for?} 

\item an idealized truncated figure that gets each cohort right (this is the current \fref{timeCutoff}), the idea is that it can also represent an idealized version of nowcast perceptions.

\item A cohort-description figure but without cumulating for the supp. This one does not need to bother with between-cohort statistics. That is going to be time-scaled version of an older figure. \ag{Does \fref{Cohort} do what we are looking for?} 

\end{itemize}

It's also possible to imagine realistic approaches between these two extremes, but let's put that off for later. There are methods (including by Dushoff and Park) for thinking about this at the cohort level, but not with a focus on individual variation. Maybe this is just for discussion. OR maybe we should also look at plots where we go up until a particular time and only count recovered infectors.
%\ag{Does \fref{timeCutoffV1} do what we are looking for?}


\begin{figure}
	\centering
	\includegraphics{kappa/PlotTrunc.Rout.pdf}
	\caption{
		\textbf{Distribution of case-per-case evolves over the course of outbreak.}
		Panel a: The size of the cohort infected at each rescaled time point is measured by incidence at that point.
		Panel b: The mean and standard deviation of case-per-case at each time are computed using the realized cases up to that point. 
		Early in the outbreak, the variability in case-per-case is yet to be fully revealed, as those still infectious generated the same number of secondary cases as those who had just recovered.
		As the outbreak approaches its peak, the contribution of the early infected but not yet recovered population is realized, increasing the standard deviation.
		After the peak, both the mean and the standard deviation approach 1 rapidly, regardless of \Ro.
		Panel c: The squared coefficient of variation of case-per-case $\kappa$ climbs over the course of the outbreak.
		Unlike in epidemics with \Ro~close to one, in stronger epidemics, the between-cohort component jumps after the outbreak peak as the susceptible pool depletes.
		For panels b and c, the simulation was stopped at each time point to compute the y-axis value. 
		The y-axis value at each time point was computed by assigning to each cohort the number of realized cases up to that time.
		In all panels, time units measured in the mean infectious period are again scaled relative to the peak time.
	}\flab{Naive}
\end{figure}


%\begin{figure}
%	\centering
%	\includegraphics{kappa/RcTimePlotVaryingPeakObs.Rout.pdf}
%	\caption{
%		\textbf{The expected number of secondary cases generated by each case evolves over the course of the outbreak.}
%		Panels (a-c) correspond to incidence, the mean and standard deviation of the number of cases per case,  and its squared coefficient of variation, respectively, as in \fref{timeCutoff}.
%		For panels b and c, the simulation was stopped at each time point to compute the y-axis value. 
%		The y-axis value at each time point was computed by taking into account the recovered individuals by that time.
%		More specifically, the contribution of each cohort to the overall mean and variance was weighted by the fraction of recovered individuals in the cohort.
%	}\flab{timeCutoffV1}
%\end{figure}

\tg{Can we make a note about for epidemics with large R0, if you don't start tracking cases right from the beginning, you'll already underestimate cases/case }
\jd{Yes, this should go into the paper.}

\section{Box (or appendix?) Tapan's
proof?}
\section{Discussion}

