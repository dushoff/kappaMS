\section{Introduction}

\section{Box: Kappa tutorial}

\section{Results}

Demographic stochasticity can generate ``emergent'' heterogeneity even in the absence of explicit differences between individual-based rates.  In simple models, this heterogeneity can be characterized. We explicate the notion that this is predictable (see Box). \jd{Is that really what Box is doing, though? Or more about linking the emergent stochasticity in the deterministic vs.\ demographic-stochastic models?}

\begin{figure}
\centering
\includegraphics{lsFig.Rout.pdf}
\caption{
	\textbf{Heterogeneity emerges even from a simple, linearized compartmental model} due to implicit variation in recovery times among infectors. (left) Activity distributions (density curves) and secondary case distributions (density histograms) for the outset of an SIR epidemic. Because the first bin (at zero) sits at the boundary of support for each distribution, we have plotted this bin as double the density and half the width; this adjustment preserves area-to-area correspondence with the PDF, while facilitating visual comparison of the heights of the density and mass functions. (right) Inequality curves for \emph{activity} distributions from SIR models with differing \Ro~are identical (and indistinguishable due to overplotting); inequality in the \emph{case} distribution decreases with R0 towards the theoretical limit of the activity distribution.
}\flab{ls}
\end{figure}

\Fref{ls} shows patterns of emergent heterogeneity in the early stage of an outbreak.

\begin{figure}
\centering
\includegraphics{rc/rcHist.Rout.pdf}
\caption{
	Some histograms. Look at poster text and see what we think.
}\flab{rcHist}
\end{figure}

But despite differences in a non-dynamic world, we find invariance in case-per-case when looking across the entire epidemic. \fref{rcHist} shows realized distributions of ``offspring cases'' caused by individual infectors across a simple, stochastic SIR epidemic. The distributions remain indistinguishable across a wide range of the key parameter \Ro.

\begin{figure}
\centering
\includegraphics{kappa/stackbar.Rout.pdf}
\caption{
	Some more histograms. 
}\flab{stackbar}
\end{figure}

This seems surprising. The resolution is that larger epidemics with larger \Ro\ have larger between-cohort variation, as expected, but that is balanced by smaller within-cohort variation. 

\begin{figure}
\centering
\includegraphics{kappa/RcTimePlotVaryingPeak.Rout.pdf}
\caption{
	How components of variance are changing through time
}\flab{timeCutoff}
\end{figure}

We are also interested in what emergent distributions will look like to people studying outbreaks in real time. We are interested, at least to some extent, both in how cohorts change through time, and in what the outbreak will ``look like'' if we observe from a particular time.  @fig:timeCutoff is one example; we are working on others.

Observing from a particular time can be done in two ways: either naively, or by trying to correct for the truncation of observations.  These can be simulated, respectively, by either simply stopping the simulation at a certain time (or reporting what would be seen if we did), or in an idealized world, by looking at all the cohorts infected up until a given time. It's worth looking at some pictures of both of these views and seeing what we think. It may also be worth looking at statistics for individual cohorts (I guess this is a bit boring, because we only have within-cohort variation in that case, but we should do it and put in the supp).

It's also possible to imagine realistic approaches between these two extremes, but let's put that off for later. There are methods (including by Dushoff and Park) for thinking about this at the cohort level, but not with a focus on individual variation. Maybe this is just for discussion. OR maybe we should also look at plots where we go up until a particular time and only count recovered infectors.

\tg{Can we make a note about for epidemics with large R0, if you don't start tracking cases right from the beginning, you'll already underestimate cases/case}
\jd{Yes, this should go into the paper.}

\section{Box (or appendix?) Tapan's
proof?}

\section{Discussion}

