% Options for packages loaded elsewhere
\PassOptionsToPackage{unicode}{hyperref}
\PassOptionsToPackage{hyphens}{url}
%
\documentclass[
]{article}
\usepackage{amsmath,amssymb}
\usepackage{iftex}
\ifPDFTeX
  \usepackage[T1]{fontenc}
  \usepackage[utf8]{inputenc}
  \usepackage{textcomp} % provide euro and other symbols
\else % if luatex or xetex
  \usepackage{unicode-math} % this also loads fontspec
  \defaultfontfeatures{Scale=MatchLowercase}
  \defaultfontfeatures[\rmfamily]{Ligatures=TeX,Scale=1}
\fi
\usepackage{lmodern}
\ifPDFTeX\else
  % xetex/luatex font selection
\fi
% Use upquote if available, for straight quotes in verbatim environments
\IfFileExists{upquote.sty}{\usepackage{upquote}}{}
\IfFileExists{microtype.sty}{% use microtype if available
  \usepackage[]{microtype}
  \UseMicrotypeSet[protrusion]{basicmath} % disable protrusion for tt fonts
}{}
\makeatletter
\@ifundefined{KOMAClassName}{% if non-KOMA class
  \IfFileExists{parskip.sty}{%
    \usepackage{parskip}
  }{% else
    \setlength{\parindent}{0pt}
    \setlength{\parskip}{6pt plus 2pt minus 1pt}}
}{% if KOMA class
  \KOMAoptions{parskip=half}}
\makeatother
\usepackage{xcolor}
\usepackage{graphicx}
\makeatletter
\def\maxwidth{\ifdim\Gin@nat@width>\linewidth\linewidth\else\Gin@nat@width\fi}
\def\maxheight{\ifdim\Gin@nat@height>\textheight\textheight\else\Gin@nat@height\fi}
\makeatother
% Scale images if necessary, so that they will not overflow the page
% margins by default, and it is still possible to overwrite the defaults
% using explicit options in \includegraphics[width, height, ...]{}
\setkeys{Gin}{width=\maxwidth,height=\maxheight,keepaspectratio}
% Set default figure placement to htbp
\makeatletter
\def\fps@figure{htbp}
\makeatother
\setlength{\emergencystretch}{3em} % prevent overfull lines
\providecommand{\tightlist}{%
  \setlength{\itemsep}{0pt}\setlength{\parskip}{0pt}}
\setcounter{secnumdepth}{-\maxdimen} % remove section numbering
\ifLuaTeX
  \usepackage{selnolig}  % disable illegal ligatures
\fi
\IfFileExists{bookmark.sty}{\usepackage{bookmark}}{\usepackage{hyperref}}
\IfFileExists{xurl.sty}{\usepackage{xurl}}{} % add URL line breaks if available
\urlstyle{same}
\hypersetup{
  hidelinks,
  pdfcreator={LaTeX via pandoc}}

\author{}
\date{}

\begin{document}

\hypertarget{introduction}{%
\section{Introduction}\label{introduction}}

\hypertarget{box-kappa-tutorial}{%
\section{Box: Kappa tutorial}\label{box-kappa-tutorial}}

\hypertarget{results}{%
\section{Results}\label{results}}

Demographic stochasticity can generate ``emergent'' heterogeneity even
in the absence of explicit differences between individual-based rates.
In simple models, this heterogeneity can be characterized. We explicate
the notion that this is predictable (see Box). {[}{[}JD: is that really
what Box is doing, though? Or more about linking the two scales of
heterogeneity?{]}{]}.

\begin{figure}
\centering
\includegraphics{lsFig.Rout.pdf}
\caption{\textbf{Heterogeneity emerges even from a simple, linearized
compartmental model} due to implicit variation in recovery times among
infectors. (left) Activity distributions (density curves) and secondary
case distributions (density histograms) for the outset of an SIR
epidemic. Because the first bin (at zero) sits at the boundary of
support for each distribution, we have plotted this bin as double the
density and half the width; this adjustment preserves area-to-area
correspondence with the PDF, while facilitating visual comparison of the
heights of the density and mass functions. (right) Inequality curves for
\emph{activity} distributions from SIR models with differing \Ro~are
identical (and indestinguishable due to overplotting); inequality in the
\emph{case} distribution decreases with R0 towards the theoretical limit
of the activity distribution.}
\end{figure}

\begin{figure}
\hypertarget{Fig:rcHist}{%
\centering
\includegraphics{rc/rcHist.Rout.pdf}
\caption{Some histograms. Look at poster text and see what we think
{[}Fig:rcHist{]}.}\label{Fig:rcHist}
}
\end{figure}

But despite differences in a non-dynamic world, we find invariance in
case-per-case when looking across the entire epidemic. Fig:rcHist shows
realized distributions of ``offspring cases'' caused by individual
infectors across a simple, stochastic SIR epidemic. The distributions
remain indistinguishable across a wide range of the key parameter
\(\Ro\).

This seems surprising. The resolution is that larger epidemics with
larger \(\Ro\) have larger between-cohort variation, as expected, but
that is balanced by smaller within-cohort variation. We claim: different
relative contributions of variability from between and within-cohorts
across R0 (bottom C of legacy/figures/emergentHetPoster.pdf). JD-Azadeh:
can you work on putting some code into kappa/ (the kappaCode repo) that
can do this?

\begin{figure}
\centering
\includegraphics{legacy/outputs/RcTimePlotVaryingPeak.Rout.pdf}
\caption{How components of variance are changing through time
fig:timeCutoff}
\end{figure}

We are also interested in what emergent distributions will look like to
people studying outbreaks in real time. We are interested, at least to
some extent, both in how cohorts change through time, and in what the
outbreak will ``look like'' if we observe from a particular time.
@fig:timeCutoff is one example; we are working on others.

Observing from a particular time can be done in two ways: either
naively, or by trying to correct for the truncation of observations.
These can be simulated, respectively, by either simply stopping the
simulation at a certain time (or reporting what would be seen if we
did), or in an idealized world, by looking at all the cohorts infected
up until a given time. It's worth looking at some pictures of both of
these views and seeing what we think. It may also be worth looking at
statistics for individual cohorts (I guess this is a bit boring, because
we only have within-cohort variation in that case, but we should do it
and put in the supp).

It's also possible to imagine realistic approaches between these two
extremes, but let's put that off for later. There are methods (including
by Dushoff and Park) for thinking about this at the cohort level, but
not with a focus on individual variation. Maybe this is just for
discussion. OR maybe we should also look at plots where we go up until a
particular time and only count recovered infectors.

TG: Can we make a note about for epidemics with large R0, if you don't
start tracking cases right from the beginning, you'll already
underestimate cases/case JD: Yes, this should go into the paper.

\hypertarget{box-or-appendix-tapans-proof}{%
\section{Box (or appendix?) Tapan's
proof?}\label{box-or-appendix-tapans-proof}}

\hypertarget{discussion}{%
\section{Discussion}\label{discussion}}

\end{document}
