
%\renewcommand \thesection{S\@arabic\c@section}
%\renewcommand\thetable{S\@arabic\c@table}

%\renewcommand\theequation{S.\@arabic\c@equation}
%AA partly adapted from main.tex on the Overleaf
\section{Materials and Methods}
The values of secondary cases in the left panel of \fref{ls},  were generated by computing the  geometric probability density function with mean \Ro~at the points $0,1,\ldots,30$.
As for the expected infectiousness, we evaluated
the exponential probability density function with mean \Ro~at $300$ equally distanced point in the
 interval $[0,30]$.

In the right panel of \fref{ls}, we used the Lloyd-Smith's approach to compute the inequality in the activity distribution. 
More specifically, we used the relation $F_{\text{trans}}(x)=\frac{1}{\Ro^2}\int_{0}^x ve^{-\frac{1}{\Ro}}dv$ which is the fraction of cases
due to those caused up to $x$ secondary cases.
The fraction of cases due to those caused more than $x$ cases would be $1-F_{\text{trans}}(x)$, which is equal to $(1+\frac{x}{\Ro})e^{-\frac{x}{\Ro}}$.
The population fraction of the individuals infected more than $x$ cases is $e^{-\frac{v}{\Ro}}$.
We used a similar approach to compute the 
 inequality in the secondary case distribution: We used the relation $F_{\text{trans}}(x)=\frac{1}{\Ro}\sum_{v=0}^x vG(v,\Ro)$ which is the fraction of cases
due to those caused up to $x$ secondary cases.
Here $G(v,\Ro)$ is a geometric distribution with mean \Ro.
The fraction of cases due to those caused more than $x$ cases equals $1-F_{\text{trans}}(x)$. 
The population fraction of these individuals would be $1 - P(x)$, where $P(x)$ is the cumulative distribution function of a geometric distribution with mean \Ro~evaluated at $x$.

To generate the top panel of \fref{RcAverage}, we used the individual based simulation and computed the case per case reproductive number in a population of size $10^4$. 
The simulation was initiated with one case.

We used a deterministic SIR model to compute the mean and variance of \Rx{c}.
First, we scaled time by the mean infectious period, so the resulting SIR model then depends on one parameter: \Ro.
The SIR differential equations were numerically solved for the proportions of susceptible $x(t)$ and infectious $y(t)$ at each time point $t$. 
The time interval for integration was set to $[0,100]$, well after the outbreak died out.
We then used the \textbf{R} function \fun{approxfun} to construct a function,  ${X}(t)$, that linearly interpolates the time-series $(t,\Ro x(t))$.
We associated a cohort to each of the first $60\%$ time points in the time-series $t$ (We used a $60\%$ cutoff to ensure cohorts had almost recovered by the end of the simulation period) and calculated the cohort-specific mean  and variance.
More specifically, for the cohort infected at time point $\tau$, the
mean, $\mu(\tau)$, and variance $\sigma^2(\tau)$ of \Rx{c} read as:
\begin{align*}
 \mu(\tau) &=  \int_{t>\tau} f(t-\tau) \int_{\tau}^{t} \Ro x(s) \ds \dt, \\
 \E[\Rx{c}^2(\tau)] &= \int_{t>\tau} f(t-\tau) \left(\int_{\tau}^{t} \Ro x(s) \ds \right)^2\dt,\\
  \sigma^2(\tau) &=  \E[\Rx{c}^2(\tau)] - \mu^2(\tau).
\end{align*}

To calculate these quantities for the cohort infected at $\tau$, the interpolating function $X(t)$ was integrated from $\tau$ to $t$, which is the case reproductive potential associated with the cohort fraction recovered at time point $t$.
The mean \Rx{c} for each cohort $\mu(\tau)$ was then the integral of the case reproductive potential weighted by the infectious period density function.
We used the same approach to compute $E[\Rx{c}^2(\tau)]$ and the variance of \Rx{c} for each cohort $\sigma^2(\tau)$.
In \fref{Cohort}, the middle panel was generated using this approach.

We obtained the mean of \Rx{c} by integrating the cohort-specific mean $\mu(\tau)$ against the incidence, $i(t) = \Ro x(t)y(t)$, and normalizing the result by the final size, $\int i(t) \dt$.
We computed the between variance by integrating over the cohort's mean $\mu(\tau)$ weighted by the incidence. 
The result was then normalized by the final size.
The with-in cohort variance  was computed by taking integral over the cohorts' variance $\sigma^2(\tau)$ weighted by the incidence.
The result was also normalized by the final size.
The total variance was the sum of the between and with-in variances. 
We also calculated the total variance independently; 
  we integrated $\E[\Rx{c}^2(\tau)]$ weighted by the incidence and divided it by the final size. 
 The result minus the squared mean of \Rx{c}~yielded the total variance.
Both approaches produced the same value for the total variance.

 All integrations were done using \fun{lsoda} method with the \textbf{R} package \pkg{deSolve}.
  All simulations were carried out using \textbf{R} \texttt{4.5.2}.
    Code for all numerical simulations is housed at: \url{https://github.com/dushoff/kappaCode}.   

We solved all integrals across a range of values for $\Ro$, using the starting values $y_0 = 10^{-9};x_0 = 1-10^{-9}$  to represent the limiting case in which there are no exogenous cases.
%, and therefore the mean case reproduction number over the course of the epidemic must be 1.
 In building these simulations, we used a range of time step sizes, noting convergence towards known and conjectured values (e.g., epidemic final size, mean case reproduction number, variance in case reproduction number) as resolution increased. 
\newpage
\section{Supplementary Materials}
\renewcommand{\thefigure}{S\arabic{figure}}
\setcounter{figure}{0}
\pagenumbering{arabic}% resets `page` counter to 1
\renewcommand*{\thepage}{S\arabic{page}}
\makeatother
\begin{figure}
	\centering
	\includegraphics{kappa/CohortPlot.Rout.pdf}
	\caption{
		\textbf{Between-cohort variation in case-per-case increases as \Ro~increases.} 
	Panel a depicts cohort size as a function of the time of infection, in units of the mean infectious period divided by the incidence peak time; the cohort size equals the incidence.
		In Panel b, the mean and standard deviation of case-per-case associated with each cohort are plotted.	
%	In an
%	epidemic with \Ro~close to 1, there is little susceptible depletion well after the outbreak onset, and the variation in infectious period mainly drives
%	the variation in \Rx{c}.
	Early in the outbreak, the susceptible pool shrinks slowly, and differences in recovery time mainly cause variations in case numbers. As a result, the average for cases is about the same as in a simple linearized model.
	Near the outbreak's peak, faster susceptible loss partly compensates for the inequality stemming from the infectious period, and, in turn, the squared coefficient of variation $\kappa$
	decreases (Panel c).
	After the peak, once the susceptible pool has become nearly constant again, variation in the infectious period once more drives variation within cohorts; $\kappa$ approaches one.
	In stronger outbreaks (higher \Ro), it takes longer (in rescaled time units) for the susceptible population to stabilize, explaining the slower move of $\kappa$ toward one.
	}
	\flab{Cohort}
\end{figure}


