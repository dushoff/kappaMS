
%\renewcommand \thesection{S\@arabic\c@section}
%\renewcommand\thetable{S\@arabic\c@table}

%\renewcommand\theequation{S.\@arabic\c@equation}
%AA partly adapted from main.tex on Overleaf
\section{Materials and Methods}

To generate the top panel of \fref{RcAverage}, we used the individual based simulation and computed the case per case reproductive number in a population of size $10^4$. 
The simulation was initiated with one case.

To compute the mean and variance of \Rx{c}, we used a  deterministic SIR model.
First, we scaled time by the mean infectious period, so the resulting SIR model then depends on one parameter \Ro, and the mean infectious period in the scaled model is one.
The SIR differential equations were numerically solved for the proportions of susceptible $x(t)$ and infectious $y(t)$ at each time point $t$. 
The time interval for integration was set to $[0,100]$; well after the outbreak died out.
We then used the \textbf{R} function \fun{approxfun} to construct a function,  ${X}(t)$, that linearly interpolates the time-series $(t,\Ro x(t))$.
We associated a cohort to each of the first $60\%$ time points in the time-series $t$ (We used a $60\%$ cutoff to ensure cohorts had almost recovered by the end of the simulation period) and calculated the cohort-specific mean  and variance.
More specifically, for cohort infected at time point $\tau$, the
mean, $\mu(\tau)$, and variance $\sigma^2(\tau)$ of \Rx{c} read as:
\begin{align*}
 \mu(\tau) &=  \int_{t>\tau} f(t-\tau) \int_{\tau}^{t} x(s) \ds \dt, \\
  \sigma^2(\tau) &=  \int_{t>\tau} f(t-\tau) \left(\int_{\tau}^{t} x(s) \ds \right)^2\dt  - \mu^2(\tau).
\end{align*}

To calculate these quantities for cohort infected at $\tau$, the interpolating function $X(t)$ was integrated from $\tau$ to obtain the case reproductive potential.
The mean \Rx{c} for each cohort $\mu(\tau)$ was then the integral of the case reproductive potential weighted by the infectious period density function.
We used the same approach to compute the variance of \Rx{c} for each cohort $\sigma^2(\tau)$.
The last two panels in \flab{Cohort} was generated using this approach.

We computed the between variance by integrating over the cohort's mean $\mu(\tau)$ weighted by the incidence $i(t) = \Ro x(t)y(t)$. 
The result was then normalized by the final size, $\int i(t) \dt$.
The with-in cohort variance  was computed by taking integral over the cohorts' variance $\sigma^2(\tau)$ weighted by the incidence.
The result was also normalized by the final size.
The total variance was the sum of the between and with-in variances. 
We also calculated the total variance independently; we first
 integrated  product of the infectious period density function and the squared of the integral $X(t)$ for each cohort.
 We then reintegrated the result weighted by the incidence and normalized by the final size.
Both approaches yield the same value for the total variance.

 All integrations were done using \fun{lsoda} method with the \textbf{R} package \pkg{deSolve}.
  All simulations were carried out using \textbf{R} \texttt{4.5.2}.
    Code for all numerical simulations is housed at: \url{https://github.com/dushoff/kappaCode}.   

We solved all integrals across a range of values for $\Ro$, using the starting values $y_0 = 10^{-9};x_0 = 1-10^{-9}$  to represent the limiting case in which there are no exogenous cases.
%, and therefore the mean case reproduction number over the course of the epidemic must be 1.
 In building these simulations, we used a range of time step sizes, noting convergence towards known and conjectured values (e.g., epidemic final size, mean case reproduction number, variance in case reproduction number) as resolution increased. 
\newpage
\section{Supplementary Materials}
\renewcommand{\thefigure}{S\arabic{figure}}
\setcounter{figure}{0}
\pagenumbering{arabic}% resets `page` counter to 1
\renewcommand*{\thepage}{S\arabic{page}}
\makeatother
\begin{figure}
	\centering
	\includegraphics{kappa/CohortPlot.Rout.pdf}
	\caption{
		\textbf{As \Ro~increases the variation in \Rx{c} shifts from within to between cohorts.}  
		In an
		epidemic with \Ro~close to 1, there is little susceptible depletion well after the outbreak inset, and the with-in variation in \Rx{c} is larger compared to an stronger epidemic and is nearly equal to the variance of the infectious period distribution.
		After the time of peak incidence, the \Rx{c} decreases.
	}
	\flab{Cohort}
\end{figure}



